\documentclass{article} % For LaTeX2e
\usepackage{iclr2022_conference,times}
% Optional math commands from https://github.com/goodfeli/dlbook_notation.
\input{math_commands.tex}

%######## APS360: Uncomment your submission name
\newcommand{\apsname}{Project Proposal}
%\newcommand{\apsname}{Progress Report}
%\newcommand{\apsname}{Final Report}

%######## APS360: Put your Group Number here
\newcommand{\gpnumber}{4}

\usepackage{hyperref}
\usepackage{url}
\usepackage{graphicx}
\usepackage{float}
\restylefloat{table}

%######## APS360: Put your project Title here
\title{Autonomous Car Starting Code Planning\\}


%######## APS360: Put your names, student IDs and Emails here
\author{Rudra Dey  \\
Student\# 1010124866\\
\texttt{rudra.dey@mail.utoronto.ca } \\
}

% The \author macro works with any number of authors. There are two commands
% used to separate the names and addresses of multiple authors: \And and \AND.
%
% Using \And between authors leaves it to \LaTeX{} to determine where to break
% the lines. Using \AND forces a linebreak at that point. So, if \LaTeX{}
% puts 3 of 4 authors names on the first line, and the last on the second
% line, try using \AND instead of \And before the third author name.

\newcommand{\fix}{\marginpar{FIX}}
\newcommand{\new}{\marginpar{NEW}}

\iclrfinalcopy 
%######## APS360: Document starts here
\begin{document}

\maketitle

\begin{abstract}
This document presents our team's procedure to create the baseline model
code for an autonomous self-driving car. ...\\

\end{abstract}

\section{Setting Up the Enironment}

\subsection{Installing CARLA}

\subsection{Installing Anaconda}


\section{Data Collection}

\subsection{Creating the Data Collection Script}

In CARLA, there are already cars which have an autonomous functionality.
We can create data a RHB Camera Sensor attached to this car, while it drives
around a designated map. 

Procedure:
\begin{enumerate}
  \item{Spawn in a vehicle}
  \item{Attach an RGB Camera Sensor (front-facing) to the dash of the car.}
  \item{Collect image data (20 fps) and steering values for the car.}
\end{enumerate}

\subsection{Collecting Data in Stable Conditions}

Initially, data would be collected without traffic in the roads (i.e
no pedestrians or cars on the road). The car would have a constant speed
and the camera will always be mounted in the same position and angle at
the front of the car. The FPS will be set to a constant 20 FPS.
We will use the CARLA's autopilot as the ground truth for our model.



\subsection{Cleaning the Data and Saving Dataset}


 
\section{Creating Baseline Model}

\subsection{Building the model}

\subsection{Evaluating the model}



\section{Training Neural Netowrk}

*Quantitative Results: Training Loss (MSE) and Validation Loss

\subsection{Model Training Script}

\subsection{Logging}

\subsection{Outputting Results}



\section{Evaluating with Inference}

*Qualitative Results: Using model for actual driving in CARLA

\subsection{Running model in CARLA using live camera feed}


\end{document}
